\documentclass[letterpaper,openany]{book}
%% \documentclass[border=2mm]{standalone}

\usepackage{tikz}
\usetikzlibrary{calc,chains,positioning,shapes,decorations.pathreplacing}
% \usepackage[T1]{fontenc}
\usepackage[table]{xcolor}
\usepackage{tikz}
\usetikzlibrary{calc,chains,positioning,shapes,decorations.pathreplacing}
% \usepackage[T1]{fontenc}
\usepackage[table]{xcolor}
\usepackage{caption}

\newcounter{lastbyte}



\usepackage{luamplib}
\mplibtextextlabel{enable}

\begin{document}

\begin{mplibcode}
  input metaobj
  s := 4.5cm; m := -3cm; % locates upper and lower boxes
  beginfig(1);
  % Central box
  newBox.msrmt("Measurement") "filled(true)", "fillcolor(.8white)", 
  "dx(.6s)", "framestyle(dashed evenly)";
  msrmt.c = (s, .5m); drawObj(msrmt);
  % Upper and lower boxes
  newBox.syst("System") "dx(2mm)", "dy(3mm)"; 
  newBox.model("Model") "dx(2mm)", "dy(3mm)";
  syst.c = (s, 0); model.c = (s, m);
  drawObj(syst); drawBox(model);
  % Empty circle
  ep := .5(xpart syst.w); t := xpart syst.e + ep; u := xpart syst.w - ep;
  newCircle.circ("") "circmargin(1.5mm)";
  circ.c = (t, m);
  drawObj(circ);
  % Connections
  drawarrow origin -- syst.w;
  drawarrow (u, 0) -- (u, m) -- model.w;
  drawarrow syst.e -- (t+ep, 0);
  drawarrow (t, 0) -- circ.n;
  drawarrow model.e -- circ.w;
  drawarrow circ.e -- (t+ep, m);
  % The spring (and its label)
  newEmptyBox.upper(0, 0); upper.c = (s, -.75m);
  picture lab; lab = textext("$d'$");
  nczigzag(upper)(syst) "coilwidth(2.5mm)", "coilarmA(0mm)", 
  "coilarmB(3mm)", "linearc(.4mm)", "labpic(lab)", "labdir(rt)";
  % Other labels  
  label.top("$u'$", (u, 0)); label.urt("$u$", (u, m));
  label.top("$t'$", (t, 0));
  label.top("$y$", .5(model.e+circ.w));
  label.rt("$t$", (t, ypart(.5(msrmt.s+circ.n))));
  label.top("$\epsilon$", .5[(t,m), (t+ep, m)]);
  labeloffset := .5bp;
  label.llft("\tiny$+$", circ.sw);
  label.urt("\tiny$-$", circ.ne);
  labeloffset := 3bp;
  endfig; 
\end{mplibcode}

\noindent\begin{tikzpicture}[
  start chain = mcastin going below,
  start chain = loopbackin going right,
  node distance = 0pt,
  mcastinsocket node/.style = {align=center, on chain=mcastin},
  loopbackinsocket node/.style = {align=center, on chain=loopbackin, rotate=270}
]
\node[mcastinsocket node] {IPv4 multicast};
\node[mcastinsocket node] {IPv6 multicast};
\node[loopbackinsocket node] {IPv4 loopback};
\node[loopbackinsocket node] {IPv6 loopback};

\end{tikzpicture}

\begin{center}
\noindent\begin{tikzpicture}[
    start chain = A going right,
    node distance = 0pt, % Ensure no gap between bit nodes
    byte node/.style = {draw=lightgray, minimum height=20pt, minimum width=(\textwidth)/56, text width=\dimexpr(\textwidth)/56\relax, text height=16pt, text depth=4pt, font=\ttfamily, anchor=south, align=center, inner sep=0pt, outer sep=0pt, on chain=A}
  ]
  \def\mybytestring{{@},{(},{\#},{)},{ },D,w,m,W,h,a,t,{ },1,.,0,.,0,{ },C,o,p,y,r,i,g,h,t,{ },D,a,n,i,e,l,{ },M,c,R,o,b,b,{ },2,0,2,5,{ },m,c,p,l,e,x,.,n,e,t}
  % Draw the byte nodes and indices
  \setcounter{lastbyte}{0}
  \foreach \byte [count=\i] in \mybytestring {
    \node[byte node, fill=white] (byte-\i) {\byte};
    \stepcounter{lastbyte}
  }
  \node[byte node, fill=lightgray!50] (termnull) {$\emptyset$};
  
  % and a label for the byte nodes
  \ifnum \value{lastbyte} > 0
    \draw [decorate, decoration={brace, amplitude=10pt}] (byte-1.north west) -- (byte-\thelastbyte.north east)
    node [above=8pt, midway, font=\scriptsize] {\texttt{view()}};
    \draw [decorate, decoration={brace, mirror, amplitude=10pt}] (byte-1.south west) -- (byte-4.south east)
    node [below=8pt, midway, font=\scriptsize] {\texttt{nth(0)}};
    \draw [decorate, decoration={brace, mirror, amplitude=10pt}] (byte-6.south west) -- (byte-12.south east)
    node [below=8pt, midway, font=\scriptsize] {\texttt{nth(1)}};
    \draw [decorate, decoration={brace, mirror, amplitude=10pt}] (byte-14.south west) -- (byte-18.south east)
    node [below=8pt, midway, font=\scriptsize] {\texttt{nth(2)}};
    \draw [decorate, decoration={brace, mirror, amplitude=10pt}] (byte-20.south west) -- (byte-47.south east)
    node [below=8pt, midway, font=\scriptsize] {\texttt{nth(3)}};
    \draw [decorate, decoration={brace, mirror, amplitude=10pt}] (byte-49.south west) -- (byte-58.south east)
    node [below=8pt, midway, font=\scriptsize] {\texttt{nth(4)}};

  \draw [decorate, decoration={brace, mirror, amplitude=10pt}] (byte-5.south west) -- (byte-5.south east) 
  node [below=8pt, midway, rotate=270, anchor=east, xshift=3em, yshift=.5pt, font=\scriptsize] {delim};
  \draw [decorate, decoration={brace, mirror, amplitude=10pt}] (byte-13.south west) -- (byte-13.south east) 
  node [below=8pt, midway, rotate=270, anchor=east, xshift=3em, yshift=.5pt, font=\scriptsize] {delim};
  \draw [decorate, decoration={brace, mirror, amplitude=10pt}] (byte-19.south west) -- (byte-19.south east) 
  node [below=8pt, midway, rotate=270, anchor=east, xshift=3em, yshift=.5pt, font=\scriptsize] {delim};
  \draw [decorate, decoration={brace, mirror, amplitude=10pt}] (byte-48.south west) -- (byte-48.south east) 
  node [below=8pt, midway, rotate=270, anchor=east, xshift=3em, yshift=.5pt, font=\scriptsize] {delim};

  \fi
\end{tikzpicture}
\captionof{figure}{\texttt{Dwm::What::SegmentedLiteral} layout}
\end{center}


\chapter{Inputs and Outputs}

\begin{tabular}{lll}
  & files & multicast \\
  loopback & yes & yes \\
  multicast & yes & no \\
\end{tabular}

\section{Inputs}
\subsection{LoopbackReceiver}
\subsubsection{IPv4 loopback, UDP}
\subsubsection{IPv6 loopback, UDP}

\subsection{MulticastReceiver}
\subsubsection{IPv4 multicast}
\subsubsection{IPv6 multicast}

\section{Outputs}
\subsection{FileLogger}
\subsubsection{files}

\subsection{MulticastSender}
\subsubsection{IPv4 multicast}
\subsubsection{IPv6 multicast}

\section{Input to output routing}
\subsection{Loopback to files}
\subsection{Loopback to multicast}
\subsection{Multicast to files}


\end{document}
