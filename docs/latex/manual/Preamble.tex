\usepackage{emoji}
\usepackage{lmodern}
\setemojifont{Apple Color Emoji}
\setmainfont{Latin Modern Sans}
\usepackage{fontspec}
\usepackage{newunicodechar}
% \usepackage{capt-of}
\usepackage{caption}
\usepackage{enumitem}
\usepackage[en-US,showzone]{datetime2}
\usepackage{fancyvrb}
% \fvset{frame=single,framesep=1mm,fontfamily=courier,fontsize=\scriptsize,numbers=left,framerule=.3mm,numbersep=1mm,commandchars=\\\{\}}
\fvset{frame=none,framesep=1mm,numbers=none,commandchars=\\\{\}}
\usepackage{fvextra}
\usepackage{float}
\usepackage{mdframed}
\usepackage{graphicx}
\usepackage{hyperref}
\usepackage{listings}
\usepackage{makecell}
\usepackage{multirow}
\usepackage{setspace}
\usepackage{tabularx}
\usepackage{titlesec}
\usepackage{titletoc}
\usepackage{tikz}
\usetikzlibrary{calc,chains,positioning,shapes,decorations.pathreplacing}
% \usepackage[T1]{fontenc}
\usepackage[table]{xcolor}
% \usepackage[scaled]{beramono}
% \renewcommand{\ttdefault}{beramono}
% \renewcommand{\familydefault}{\sfdefault}

\directlua{luaotfload.add_fallback
   ("emojifallback",
    {
      "AppleColorEmoji:mode=harf;"
    }
   )}

\setmonofont{Fira Code}[
  RawFeature={fallback=emojifallback}
]

\titleformat{\chapter}[display]
            {\normalfont\huge\bfseries}
            {\chaptertitlename~\thechapter}{20pt}{\huge}

\titlespacing*{\section}{0pt}{1ex}{.1ex}
\titlespacing*{\chapter}{0pt}{1ex}{.1ex}
\titlespacing*{\subsection}{0pt}{1ex}{1ex}
\titlespacing*{\subsubsection}{0pt}{1ex}{1ex}

\titlecontents{chapter}   % Affects chapters
[0em]                    % Left indentation
{}                       % Space before entry
{\bfseries\contentslabel{2.3em}}  % Font for numbered chapters
{}                       % Font for unnumbered chapters
{\titlerule*[1pc]{.}\contentspage}    % Dot leaders
{}          % Page number
\titlecontents{section}   % Affects sections
[1em]                    % Left indentation
{}                       % Space before entry
{\bfseries\contentslabel{2.3em}}  % Font for numbered sections
{}                       % Font for unnumbered sections
{\titlerule*[1pc]{.}\contentspage}    % Dot leaders
{}          % Page number
\titlecontents{subsection}   % Affects subsections
[2.5em]                    % Left indentation
{}                       % Space before entry
{\contentslabel{3.3em}}         % Font for numbered subsections
{}                       % Font for unnumbered subsections
{\titlerule*[1pc]{.}\contentspage}    % Dot leaders
{}          % Page number
\titlecontents{subsubsection}   % Affects subsubsections
[1em]                    % Left indentation
{}                       % Space before entry
{\bfseries\contentslabel{2.5em}}      % Font for numbered subsubsections
{\bfseries\contentslabel{2.5em}}      % Font for unnumbered subsubsections
{\titlerule*[1pc]{.}\contentspage}    % Dot leaders
{}          

\titleformat{\subsubsection}
  {\normalfont\large\bfseries}{\underline{\thesubsubsection}}{1em}{}
  
\textwidth=6.5in
\textheight=9.0in
\oddsidemargin=-0.25in
\evensidemargin=-0.25in
\topmargin=-0.25in
\parskip=1em
\parindent=0in
\itemsep=.1ex

% the styles for short and long nodes
\tikzset{
short/.style={draw,text height=3pt,text depth=20pt,
  text width=7pt,align=center,fill=blue!10},
shortTall/.style={draw,text height=1pt,text depth=36pt,inner sep=1pt,
  text width=.013888\textwidth, align=center,fill=blue!10},
long/.style={short,fill=green!10,text width=(\textwidth-64pt)/3},
nineTall/.style={long,fill=green!10,text width=(\textwidth)/9,
  align=center,text centered,text depth=20pt,minimum height=36pt,
  text height=18pt},
dotdot/.style={fill=white,minimum width=(\textwidth-96pt)/3},
eightByteEmpty/.style={draw,fill=blue!10,text depth=20pt,text width=96pt,align=center},
fourByteEmpty/.style={draw,fill=blue!10,minimum height=20pt,text depth=10pt,text width=48pt,align=center,text centered}}

\makeatletter
    \let\pgf@decorate@@brace@brace@code@old\pgf@decorate@@brace@brace@code
    \def\pgf@decorate@@brace@brace@code{
        \ifdim\pgfdecoratedremainingdistance<4\pgfdecorationsegmentamplitude
            \pgftransformxscale{\pgfdecoratedremainingdistance/4\pgfdecorationsegmentamplitude}
            \pgfdecoratedremainingdistance=4\pgfdecorationsegmentamplitude
        \fi
        \pgf@decorate@@brace@brace@code@old
    }
    \makeatother

% Using Courier font
% \renewcommand{\ttdefault}{pcr}

\definecolor{dwmOrange}{RGB}{255,127,42}
\definecolor{codegray}{rgb}{0.7,0.7,0.7}
\definecolor{codeblue}{rgb}{0.2,0.2,1.0}
\definecolor{darkblue}{rgb}{0.0,0.0,0.7}
\definecolor{green1}{RGB}{0,160,0}
\definecolor{red1}{RGB}{160,0,0}
\definecolor{numberbg}{rgb}{0.9,0.9,0.2}
\definecolor{numbercolor}{rgb}{0.2,0.2,1.0}
\colorlet{mynumberbg}{numberbg!20}
\colorlet{mynumbercolor}{numbercolor!40}

\colorlet{delimColor}{red}
\colorlet{typeColor}{codeblue}
\colorlet{statusColor}{green1}
\colorlet{nameColor}{magenta}
\colorlet{versionColor}{violet}
\colorlet{copyrightColor}{brown}
\colorlet{otherColor}{dwmOrange}

\lstdefinestyle{mystyle}%
{basicstyle=\ttfamily,
 keywordstyle=\ttfamily\color{black},
 identifierstyle=\ttfamily\color{black},
 morekeywords={std::string, cout, cin, vector, map, assert},
 keepspaces=true,
 showspaces=false,                
 showstringspaces=false,
% numberstyle=\sffamily\small\color{codegray},
 numbers=left,                    
 numbersep=1em,
 frame=l,
 framesep=2em,
 framexleftmargin=.8em,
 fillcolor=\color{mynumberbg},
 rulecolor=\color{mynumberbg},
 numberstyle=\sffamily\small\color{mynumbercolor}
}

\lstset{language=C++,
  style=mystyle
}

\newcounter{lastbyte}

\DefineVerbatimEnvironment{fnverb}{Verbatim}{
  fontfamily=\ttdefault,
  fontshape=\updefault,
  fontseries=bx,
  breaklines,
  breakanywhere,
  breaksymbolleft=,
  breakindent=0pt
}

\hypersetup{
    colorlinks=true,
    linkcolor=blue,
    filecolor=magenta,    
    urlcolor=cyan,
    pdftitle={libDwm I/O},
    pdfpagemode=FullScreen,
}

\ExplSyntaxOn
\NewExpandableDocumentCommand{\numtestTF}{mmm}
 {% #1 = possibly complex test, #2 = true text, #3 = false text
  \int_compare:nTF {#1} {#2} {#3}
 }
\ExplSyntaxOff

\newcommand{\dwmNote}[1]{
  \begin{mdframed}[backgroundcolor=blue!10]
  \textbf{NOTE:} #1
  \end{mdframed}
}

\newcommand{\dwmTodo}[1]{
  \begin{mdframed}[backgroundcolor=red!10]
  \textbf{TODO:} #1
  \end{mdframed}
}

\newcommand{\whiteCheckMark}{\emoji{white-check-mark}}
